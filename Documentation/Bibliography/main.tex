\documentclass[a4paper]{article}

\input{Preamble_commands}

% Diagrams: https://mermaid.live/
% Tips for Writing Technical Papers: https://cs.stanford.edu/people/widom/paper-writing.html

\begin{document}

%%%%%%%%%%%%%%%%
% TITLE & NAME %
%%%%%%%%%%%%%%%%
\selectlanguage{english}
\title{Bibliography}
\author{Pedro Maria da Costa Almeida Martins}
\istid{99303}
\email{pedromcamartins@tecnico.ulisboa.pt}
\advisor{Prof. Paulo Oliveira}
\coadvisor{Pedro Santos}

% To make the title
\maketitle
\thispagestyle{empty}
\clearpage


%%%%%%%%%%%%%%%%%%%%%
% TABLE OF CONTENTS %
%%%%%%%%%%%%%%%%%%%%%
\tableofcontents
% \listoffigures      % List of Figures
% \listoftables       % List of Tables
% \lstlistoflistings  % List of Listings
\thispagestyle{empty}
\clearpage


%%%%%%%%%%%%%%%%
%   MAIN TEXT  %
%%%%%%%%%%%%%%%%


\section{A novel distributed architecture for unmanned aircraft systems based on Robot Operating System 2}

Good historical context of ROS and DDS. 
Example of multi-layered architecture for vision-based navigation. 
Used PX4 internal control, sending setpoints from offboard computer. 

\cite{bianchi2023novel}

Robotic software architectures aim to solve the need for scalablity, modularity and parallelism inside robotic systems. 
The Robot Operating System (ROS) was created to address this need. 
However, it does not satisfy real-time constraints and is not natively fault tolerant. 
ROS2 addresses these issues, focusing on crossplatform, soft real-time systems. 
To satisfy soft real-time systems, new middleware technologies were developed. 
DDS has been adopted as the default middleware for ROS2, allowing developers to use Quality of Service configuration (deadlines, reliability, durability), and support scalability, with no overhead [discussed in a later section of the paper]. 

Today, electronics for drones are composed of several SoCs, depending on the complexity of the system. 
It is usual to delegate low-level control algorithms to small, real-time SoCs, while high-level algorithms are run on more powerful systems. 
Therefore, the problem of establishing a connection between these two systems arises. 
DDS is one solution for this problem. 

The Application of Autonomous Quadcoper was used, following a layered architecture, composed of modules (ROS2 nodes). 
- Lowest row: soft real-time modules 
- Middle row: high-level, slower algorithms 
- Top row: supervision logic 

---

Suggestion for out system

\paragraph{Soft real-time modules: Flight Control. }
It offers ROS2 services that enable other modules to resquest flight operations: 
- mode services: arm, disarm, offboard

\paragraph{Soft real-time modules: Sensor Fusion. }
It is responsible for fusing data from the IMU and GPS. 

\paragraph{Soft real-time modules: Actuator. }
It is responsible for controlling the motors and servos. 

\paragraph{High-level modules: Controller. }
It is responsible for controlling the drone.
It uses the data from the sensor fusion module and sets the actuators accordingly.
It is composed of two loops:
- Outer loop: high-level control, responsible for the trajectory generation.
- Inner loop: low-level control, responsible for the attitude control.


\section{Design and test of electromechanical actuators for thrust vector control}

\cite{cowan1993design}


\section{A Structured Approach for Modular Design in Robotics and Automation Environments}

\cite{elkady2013structured}


%%%%%%%%%%%%%%%%
% BIBLIOGRAPHY %
%%%%%%%%%%%%%%%%
\clearpage
% add Bibliography to TOC
\addcontentsline{toc}{section}{\refname}
% The following command resets the 'emphasis' style for bibliography entries
\normalem
% Name of your BiBTeX file
\bibliography{./Bibliography}
% The following command modifies the 'emphasis' style for bibliography entries
\ULforem


%%%%%%%%%%%%%
%    END    %
%%%%%%%%%%%%%

\end{document}
